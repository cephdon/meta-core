Some aspects of TTA are not yet fully represented in the synthesized TrueTime models.  Redundant communication networks, membership services and time synchronization are important services required for robust execution and fault-tolerance.  Future work will expand the current focus of TrueTime models to include better modeling of fault conditions and the impact faults may have on controller performance.  Time synchronization between nodes is an important aspect of TTA, but is somewhat moot in Simulink/TrueTime since all nodes can rely upon the global simulation clock.  We plan on enhancing our TrueTime models to incorporate a time synchronization protocol and to allow local clocks on nodes to have some drift and error from the global simulation clock.  Together, these enhancements should allow our TrueTime models to more accurately reflect TTA behavior.

Inclusion of sporadically executed tasks into the static schedule is also an interesting research direction.  Some existing TTA-based approaches allow limited execution of tasks outside of the static task schedule.  Support for sporadic execution will require extending both our online scheduler and our communications network interfaces.

In this paper we presented an extension to the ESMoL tool chain for synthesizing TTA-based TrueTime models.  Automatic synthesis via the tool chain greatly reduces the level of effort needed to create TrueTime models.  Leveraging TrueTime for system prototyping is useful compared to generating code directly for the embedded platform because it is far easier to explore and debug execution behavior, alter model design, or tap data streams in TrueTime/Simulink than it is with C-code loaded onto an actual embedded system.  Once system behavior has been analyzed and design criteria have been met in TrueTime the final transition to deploying onto an embedded platform is eased.
The presented toolchain is the subject of ongoing research and development. The toolchain is based on the pervasive use of models in the development, and it is targeting a well-defined, robust computational platform based on the time-triggered approach. Design-time scheduling and analysis are important ingredients. The HIL environment provides rapid, realistic experimental evaluation of controller designs. In our applications we have taken advantage of the passivity-based control design techniques that further mitigate the effects of platform uncertainties.  

The original Starmac control software was written in C over approximately four years (as of this writing), including at least three iterations of the control design and platform, as well as extensive testing.  The tools presented here have been in development nearly as long. For the quad-rotor vehicle, the control design, platform customization, model-based redesign, and generation of software (up to the testable point) took approximately three man-months using these prototyping tools.

Promising research avenues, system limitations, and tool deficiencies drive our future efforts.  The controller models have already been extended to include additional features (e.g., yaw control).  Unfortunately, the current implementation is right at the edge of the processing capabilities of the Robostix AVR processor using a floating-point emulation library.  We are considering techniques which can provide validated fixed-point implementations.  Decreasing the sampling frequency is another promising avenue, but which requires serious theoretical analysis to ensure safety.
   
The modeling tools need additional development in requirements capture, as well as more detailed hardware descriptions. Better hardware descriptions will allow us to generate code for devices within the microcontrollers, and to enrich the set of communication media and protocols for which we can generate code.   In particular, platform-dependent communication concepts still need a better representation in the modeling environment.  \cite{aces08} discusses future tool directions in greater detail. 
\section{Execution Environment}

The tool suite includes a simple, portable time-triggered virtual machine\cite{timed:frodo} 
which can run on generic Linux or FreeRTOS to implement timed execution of tasks and messages.  
The virtual machine is a lightweight implementation of the time-triggered 
architecture\cite{timed:tta} (see \cite{timed:frodo} for details).  Execution requires 
configuration with the computed cyclic schedule.  Code generated for the virtual machine 
conforms to a particular structure -- each task reads its input variables, invokes its 
component functions, and writes its output variables.  Data structures describe the invocation 
times of each task and any other necessary parameters.  The configuration also includes 
time-triggered messaging.  Each message instance includes invocation time as well as local buffer 
addresses where the virtual machine should store the computed cyclic schedule.  Code generated for 
the virtual machine conforms to a particular structure -- each task reads its input variables, 
invokes its component functions, and writes its output variables.  Data structures describe the 
invocation times of each task and any other necessary parameters.  The configuration also includes 
time-triggered messaging.  Each message instance includes invocation time as well as local buffer 
addresses where the virtual machine should store data.

%In terms of the behaviors specified in the modeling language, the calculated 
% release times represent assumptions from the control design and the scheduling model.  
The virtual machine enforces timing constraints to ensure correct execution. The passive 
control design provides slack so that dynamic stability can tolerate imperfect execution 
of the tasks.  The virtual machine must provide clock synchronization to preserve dependency 
ordering and avoid collisions.

The biggest limitations in our implementation are fault intolerance and lack of support 
for executing multiple components in a task. Our virtual machine can detect local deadline 
overruns, but at this stage it simply reports them.  We use a single master frame message 
at the start of each hyperperiod to maintain synchronization, and so are not robust to 
loss of synchronization.  Testing shows that we are also not yet robust to data hazards 
such as floating point exceptions.  The lack of support for multiple components in a task 
is not a technical issue, rather we have not yet implemented the synchronous data flow 
scheduling of those components.

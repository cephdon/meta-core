\section{Introduction}
\label{S:introduction}
%Need an introduction here...
%Gabor, Janos ... ?
%Can we provide other examples for 'health indicators', 'defense optimization'
The design of resilient control systems necessitates novel developments at the 
intersection of computer science and control theory.  The control of complex 
dynamic systems is a well-studied area, but much less is known about how to 
implement such control systems that are able to tolerate shortcomings of 
non-ideal software and network-based implementation platforms. Additionally, 
not 
only implementation side-effects have to be mitigated, but 
also potential issues related to security of the control system. For instance, 
if an operator's interface is compromised, will the attacker be able to
set control gains in such a manner so as to destabilize the system?
If an actuator is degrading or has been compromised, will the operator
be able to quickly identify and quantify such a system fault?
Many processes are highly non-linear and quite difficult to control,
as a result typical linear approximations are often made and the
resulting ``safe'' operating range is quite narrow.  Interconnection
damping assignment passivity based control (IDA-PBC) is an emerging
method to systematically tackle the design of highly non-linear
systems and derive intuitive control laws typically with many linear
control-law elements and reasonable tuning gain ranges to allow an
operator more flexibility in tuning a given system and to identify system
degradation, all while being able to safely limit the operator from
accidentally introducing destabilising gains.

% Key idea:
% Passivity based control principles provide a mathematically rigorous framework to
% construct high-confidence control systems which have infinite gain margins, thus
% possessing a great deal of robustness to uncertainty.  
Passivity is a mathematical property of the controller implementation, and 
could be realized in different ways. The approach described here applies to a 
large family of physical systems  which can be described by both linear and 
non-linear system models
\cite{haddad08:_nonlin_dynam_system_and_contr,
  schaft99:_l2_gain_passiv_nonlin_contr,
  ortega98:_passiv_based_contr_euler_lagran_system}, including
systems which can be described by cascades of passive systems such as
quad-rotor aircraft
\cite{kottenstette09:_digit_passiv_attit_and_altit}.  Most recently
IDA-PBC has been shown to be effective in determining controllers which
render non-minimum phase systems (which are typically very difficult
to control) to be dissipative and asymptotically stable  
\cite{doerfler09:_introd_to_inter_and_dampin, ramirez09:_contr_of_non_linear_proces}.

A classic, but challenging control problem related to process control
is the four tank process which can exhibit both
minimum and \nonminimum-phase behavior by simply changing valve flow ratios
\cite{johansson00:_quadr_tank_proces,
  johnsen07:_inter_and_dampin_assig_passiv}.  In
\cite{johnsen07:_inter_and_dampin_assig_passiv} it is first shown how
to derive a proportional-IDA-PBC-law to control a less complex two tank
process before deriving a control law for the four tank process.  We
continue the study of the two tank process by introducing an integrator
term to account for system uncertainty while introducing an integrator
anti-windup compensator similar to that used to control thermal
systems \cite{fu10:_feedb_therm_contr_for_real_time_system}.

Section~\ref{S:control} provides an overview on IDA-PBC and an
effective way to implement an anti-windup controller in order to
improve system resilience.  Section~\ref{S:tanks} recalls the system
dynamics used to model the flow for coupled tank systems in addition
it includes the Hamiltonian chosen to generate our control laws.
Section~\ref{S:simulations} provides simulated results demonstrating
resilience to actuator degradation and improvement in reducing
oscillations with anti-windup control.  Section~\ref{S:conclusions}
provides conclusions for this paper.


\section{Related Work}
\label{section:related}

The present technique has much in common with techniques in online verification. 
Online verification is not a new concept in computational domains.  See for example 
\cite{verif:onlinerules} and \cite{verif:pathfinder} for examples which integrate formal logical 
models and executing code to detect errors.

Sammapun, Lee, and Sokolsky\cite{online:rtmac} address lightweight online verification for 
checking timeliness and reliability in real-time systems.  RT-MaC uses a stochastic approach which 
yields confidence intervals for specified properties.  This is probably the closest in practice
to our technique.  See also recent work from Zuliani, Platzer, and Clarke 
\cite{verif:bayesian} which focuses on statistically checking properties on simulated 
trajectories in Simulink models, and which should be just as applicable to traces collected
from a running control system.

% Some of the related research is more particular to digital control systems.
% ORTEGA\cite{online:ortega} is a real-time fault-tolerant guard for dynamic 
% control.  When runtime faults are detected, ORTEGA switches the system over to a simpler
% (more computationally efficient) controller to provide resources for fault handling
% tasks.  We see this as complementary to our technique, as sector search could be applied
% in a similar way.  

From the control community self-triggered control\cite{control:selftriggered} is a framework 
which could also be used for online stability guarantees.  Control actions are timed based on 
specified criteria on measured or estimated states rather than periodic sampling.  This 
suggests interesting extensions to our technique to accomodate non-uniform sampling.
Skaf and Boyd present techniques for calculating and verifying quantization levels offline
using constraint problems representing design requirements, system behavior, and 
objective functions for assessing performance\cite{opt:trunc1}. \cite{opt:trunc} describes
a technique for calculating a set from which controller coefficients can be selected without
compromising performance.  A better understanding of the sector relationships to platform
and design parameters may lead us to be able to benefit from this technique.


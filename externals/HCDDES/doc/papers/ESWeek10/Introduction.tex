\section{Introduction}

High confidence embedded systems development features multi-faceted design activities
complemented by model and data analysis to ensure the correct operation of the final
product.  Such analysis is multi-faceted due to the heterogeneity of domains 
involved in the problems (e.g. cyber-physical systems), leading to high costs and
extended schedules for development efforts.  Analyses suffer from difficulties
of scale due to state and parameter space explosion problems.  Further, interactions
between design domains (such as the physical continous-time and computational 
discrete-events) are poorly understood, often requiring analyses to be overly conservative.
Recently the research community for embedded systems has explored 
compositional techniques for design and analysis, where system correctness properties
may be inferred from component correctness properties and more abstract interaction 
models defined on the interconnection structure of the design.  Compositional methods
strive to dramatically reduce state explosion problems by assuring correctness during
model construction, while avoiding overly conservative approximations.

ESMoL is a modeling language built to experiment with encoding compositional 
techniques for high-confidence systems into a single modeling environment.  Prior 
published work on ESMoL has dealt with modeling, code generation, and deployment for distributed 
control software implemented using our realization of the time-triggered model of computation
\cite{timed:frodo, modeling:aces08, sched:analysis}. We rely on the time-triggered paradigm to 
reduce jitter in sampling and actuating control tasks, to preserve synchronous design properties 
(such as deadlock freedom), and to detect timing faults\cite{timed:tta}. The modeling tools also 
include the ability to generate platform-specific simulation models using the TrueTime toolkit for 
Simulink \cite{modeling:truetime,tools:truetime}. The execution environment addresses only part of 
the problem.  High-confidence designs require assurances of correct software behavior, including 
guarantees for stability, schedulability and deadlock freedom.  The ESMoL-centered research effort 
seeks to address such problems using compositional design paradigms, for example using passive 
control techniques to ensure closed-loop stability for the controlled system (see Section 
\ref{section:background} for an overview).

Beyond the multi-faceted challenges described above, one significant challenge lies in 
assessing the correct behavior of control system software once all of the components have
been integrated and deployed to the hardware platform.  For example, control designers use a very 
clean notion of timing behavior that is based on simple uniform sampling intervals, and which hard 
real-time implementations try to emulate. In reality, moving a control design from models to 
deployment is much more difficult in the case of distributed controllers.  Calculation uncertainties 
due to limited data precision and timing uncertainties due to network delays introduce errors into 
the control system behavior.  Commonly, control designers try to include enough 'slack' in the 
design to avoid such platform effects.  We would like to be able to assess whether a deployed 
software control component is still stable as designed, or whether platform effects have changed the 
system behavior in a way that violates our assumptions.

To meet this challenge we propose an online stability validation technique.  The specific
requirements and challenges are as follows:

\begin{enumerate}
 \item It must be efficient, both in space and in computation.  
 \item When operating, it must be sensitive enough to detect instability for various platform 
effects. In essence it should provide an abstract metric for stability in the presence of 
different possible destabilizing conditions.
 \item The analyzer should operate on sample traces collected either from simulation or from 
the running system. We would like to be able to provide meaningful comparisons of results between
simulation and the actual deployed system to aid system troubleshooting.
\end{enumerate}

Our solution is based on sector theory, proposed originally by Zames\cite{control:sectors1}
and extended by Willems\cite{control:sectorswillems} to deal with nonlinear elements in
a control design.  Sectors provide two real-valued parameters which represent bounds on the 
possible input/output behaviors of a control loop.  Kottenstette \cite{quad:passcontrol} presented a sector
analysis block for validating a control design in Simulink.  We propose to use the same
structure to verify the deployed control software online.  Note that our assessment of
item 3 above is still only preliminary.  This particular technique provides several advantages
beyond our original requirements:

\begin{enumerate}
 \item For a given component, the sector measures behavior simultaneously over multiple inputs and 
outputs, so only one sector analyzer is required per control loop.
 \item Because of our passive abstraction of the system design (described below), using multiple
sector analyzers allowed us to quickly isolate problem components in the deployed design.
\end{enumerate}


Section \ref{section:background} gives a brief overview of the our control design approach,
relevant sector analysis concepts, and describes our solution.   Section \ref{section:solution} 
introduces describes our example and models in more detail.  Section \ref{section:results} 
discusses our experimental setup, and gives a first example of a problem uncovered by the online 
sector analysis.  Section \ref{section:params} describes our preliminary findings on the 
relationships between various system parameters and the sector values.   Finally we visit 
related work (Section \ref{section:related}), and future work and conclusions 
(Section \ref{section:conc}).
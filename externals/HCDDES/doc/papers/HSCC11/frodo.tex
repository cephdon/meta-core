\section{FRODO Virtual Machine}
\label{Section:FRODO}

The ESMoL modeling language and its attendant analysis tools capture the design intent of an embedded system and provide some reassurance as to its behavior.  The effectuation of an ESMoL model into correct executable code is itself a significant challenge.  The functional code necessary to implement the software components is synthesized, but the time-triggered behavior assumed by the time-triggered architecture model of computation must also be realized.  Time-triggered execution could be supported at the operating system level, but few such systems are available \cite{ttp_os, osek_tt}.  Instead, an alternate approach is to implement a time-triggered execution layer on top of standard OS primitives \cite{hhk_01,ccmstn_03,YYY}.  As long as all underlying TTA assumptions are maintained, this approach provides a portable and light-weight platform for robust execution.  The FRODO v2 virtual machine (VM) provides time-triggered execution semantics for ESMoL models and has been integrated into the overall ESMoL toolchain.


%Ideally, the design of embedded systems would not be limited to strictly time-triggered execution semantics.  In many situations event-triggered semantics would provide a more natural and intuitive approach for designers.  In an airplane, for example, high-criticality messages for flight control should execute using time-triggered semantics, but low-criticality messages for in-flight entertainment could execute using event-triggered semantics.  The problem exists to find a robust means to blend both time-triggered and event-triggered execution while still maintaining the fault tolerance and predictability of a TTA.  The ESMoL toolchain supports the modeling and analysis of both time- and event-triggered execution, and the FRODO v2 virtual machine provides run-time support for blended execution.

%Experimentation with the FRODO v2 virtual machine is very straightforward.  The FRODO VM has support for multiple OS platforms, several different communication networks, and provides fully-integrated data logging.  Because of FRODO's tight integration with ESMoL toolchain, and the fact that the toolchain fully synthesizes all necessary code, creating an experiment using FRODO is trivial once an ESMoL model has been fully defined.  A designer can easily create a model in Simulink, annotate it in ESMoL, generate a complete FRODO-based embedded executable, and run experiments all on their individual PC.  The ease-of-use and flexibility of ESMoL with FRODO has allowed us to complete several experiments.